%
% テンプレート作ってみた.とりあえずテスト.
% \documentclass[uplatex, a4paper,dvipdfmx,10pt]{jsarticle}
%

\usepackage{nidanfloat} %% add
\usepackage{graphics}
\usepackage[dvipdfmx]{hyperref}
\usepackage{pxjahyper}
\usepackage{graphicx}
\usepackage{plain}
\usepackage{fancyhdr}
\usepackage{bm}
\usepackage{amsmath} % bmatrix
\newcommand{\argmax}{\mathop{\rm argmax}\limits}
\newcommand{\argmin}{\mathop{\rm argmin}\limits}
\usepackage{amssymb} % leqq
\usepackage{float}
\usepackage{comment} % 複数行コメント
\topmargin = -5mm
\oddsidemargin = 5mm
\textwidth = 152mm
\textheight = 240mm
% \pagestyle{empty}
\pagestyle{plain}
\renewcommand{\footrulewidth}{0pt}
\renewcommand{\headrulewidth}{0pt}
\bmdefine{\bx}{x}
\bmdefine{\by}{y}
\everymath{\displaystyle}

\usepackage{listings}
\usepackage{xcolor}

\definecolor{codegreen}{rgb}{0,0.6,0}
\definecolor{codegray}{rgb}{0.5,0.5,0.5}
\definecolor{codepurple}{rgb}{0.58,0,0.82}
\definecolor{backcolour}{rgb}{0.95,0.95,0.92}
 
\lstdefinestyle{mystyle}{
    backgroundcolor=\color{backcolour},   
    commentstyle=\color{codegreen},
    keywordstyle=\color{magenta},
    numberstyle=\tiny\color{codegray},
    stringstyle=\color{codepurple},
    basicstyle=\ttfamily\footnotesize,
    breakatwhitespace=false,         
    breaklines=true,                 
    captionpos=b,                    
    keepspaces=true,                 
    numbers=left,                    
    numbersep=5pt,                  
    showspaces=false,                
    showstringspaces=false,
    showtabs=false,                  
    tabsize=2
}
 
\lstset{style=mystyle}
